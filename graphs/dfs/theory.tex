\documentclass[a4paper,20pt]{article}
\usepackage[left=2cm,right=2cm,
top=2cm,bottom=2cm,bindingoffset=0cm]{geometry}
\usepackage{cmap}
\usepackage[T2A]{fontenc}
\usepackage{amsmath,amssymb,amsfonts,amsthm,mathtools}
\usepackage[utf8]{inputenc}
\usepackage{fancybox,fancyhdr}
\usepackage{euscript,mathrsfs}
\usepackage{float}
\DeclareMathOperator{\tg}{tg}
\DeclareMathOperator{\arctg}{arctg}
\DeclareMathOperator{\ctg}{ctg}
\DeclareMathOperator{\arcctg}{arcctg}
\DeclareMathOperator{\sh}{sh}
\DeclareMathOperator{\ch}{ch}
\DeclareMathOperator{\cth}{cth}
\DeclareMathOperator{\asinh}{asinh}
\DeclareMathOperator{\acosh}{acosh}
\DeclareMathOperator{\atanh}{atanh}
\DeclareMathOperator{\acoth}{acoth}
\DeclareMathOperator{\asech}{asech}
\DeclareMathOperator{\acsch}{acsch}
\DeclareMathOperator{\atan}{atan}
\DeclareMathOperator{\acot}{acot}
\DeclareMathOperator{\acos}{acos}
\DeclareMathOperator{\asin}{asin}
\DeclareMathOperator{\asec}{asec}
\DeclareMathOperator{\acsc}{acsc}
\DeclareMathOperator{\sech}{sech}
\DeclareMathOperator{\csch}{csch}
\usepackage[english]{babel}
\usepackage{multicol}
\usepackage{icomma}
\usepackage[usenames]{color}
\usepackage{colortbl}
\DeclareUnicodeCharacter{2212}{-}
\newtheorem{definition}{Definition}

\numberwithin{equation}{subsubsection}
\numberwithin{definition}{subsubsection}

\fancyhead[R]{Якимов Г.В., гр. 19122. Тема №4: Компоненты связности}
\fancyhead[L]{}
\fancyhead[C]{}

\fancyfoot[R]{}
\fancyfoot[L]{}
\fancyfoot[C]{}

\begin{document}
	\pagestyle{fancy}

	\noindent\textbf{Тема:} Компоненты связности (найти количество компонент связности графа, вывести компоненты списком вершин согласно нумерации в матрице смежности).
	\newline

	\noindent Для начала рассмотрим алгоритм $DFS$ поиска в глубину, на котором базируется алгоритм решения задачи. Существует множество вариантов алгоритма $DFS$, в зависимости от задачи: формирование леса поиска по исходному графу, расстановка меток времени и т.д. В данном случае нам нужно в процессе обхода графа, начав с определенной вершины, запомнить все вершины, до которых дойдет алгоритм, тем самым будет сформирована компонента связности этой вершины.
	$DFS$ использует стек для хранения вершин в процессе обхода, в данном случае будем использовать стек вызовов для этой цели, то есть алгоритм $DFS$ будет рекурсивным. Таким образом, функцию $DFS(v, visited, matrix)$ можно представить в виде следующих шагов:
	\newline
	\newline
	Пусть $G$ - неориентированный граф.
	\newline
	\{\textbf{In:} вершина $v \in V(G)$; множество $visited$ посещенных вершин (при первом вызове пустое); матрица смежности $matrix$ графа $G$\}
	\newline
	\{\textbf{Out:} множество всех вершин, до которых дошел алгоритм в процессе обхода\}

	\begin{enumerate}
		\item[(1)] Вершина $v$ добавляется в множество $visited$;
		\item[(2)] С помощью матрицы смежности обходятся все вершины, смежные с $v$;
		\item[(3)] Для очередной вершины $u$, смежной с $v$ и не лежащей в множестве $visited$, запускается: $visited = DFS(u, visited, matrix)$;
		\item[(4)] Возвращается множество $visited$ всех посещенных в процессе обхода вершин.
	\end{enumerate}

	\noindent\textbf{Описание алгоритма поиска компонент связности $connected\_components(Q, matrix)$:}
	\newline
	\{\textbf{In:} множество $Q$ всех вершин графа $G$, матрица смежности $matrix$ графа $G$\}
	\newline
	\{\textbf{Out:} множество всех компонент связности\}
	\newline
	Пусть $result$ - множество компонент связности (инициализируется пустым множеством).
	\begin{enumerate}
		\item[(1)] Из $Q$ извлекается вершина $v = Q[0]$. Запускается $component = DFS(v, \{\}, matrix)$. Функция $DFS$ полностью определяет компоненту связности $component$, в которой находится вершина $v$;
		\item[(2)] $component$ добавляется в $result$;
		\item[(3)] Из $Q$ удаляются все вершины, которые находятся в $component$;
		\item[(4)] Если $Q$ не пусто, то исполнение возвращается к шагу (1). В противном случае возвращается множество $result$.
	\end{enumerate}
	
\end{document}

